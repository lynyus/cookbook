\documentclass[%
a4paper,
twoside,
11pt
]{article}

% encoding, font, language
\usepackage[T1]{fontenc}
\usepackage[utf8]{inputenc}
\usepackage{lmodern}
\usepackage[ngerman]{babel}

\usepackage{nicefrac}

\usepackage[
%    handwritten,
    nowarnings,
    %myconfig
]
{config/xcookybooky}

\usepackage{blindtext}    % only needed for generating test text

\DeclareRobustCommand{\textcelcius}{\ensuremath{^{\circ}\mathrm{C}}}


\setcounter{secnumdepth}{1}
\renewcommand*{\recipesection}[2][]
{%
    \subsection[#1]{#2}
}
\renewcommand{\subsectionmark}[1]
{% no implementation to display the section name instead
}


\usepackage{hyperref}    % must be the last package
\hypersetup{%
    pdfauthor            = {Kathrin Welzel and Marcel Gro{\ss}mann},
    pdftitle             = {Autumn Recipes},
    pdfsubject           = {Recipes},
    pdfkeywords          = {example, recipes, cookbook, xcookybooky},
    pdfstartview         = {FitV},
    pdfview              = {FitH},
    pdfpagemode          = {UseNone}, % Options; UseNone, UseOutlines
    bookmarksopen        = {true},
    pdfpagetransition    = {Glitter},
    colorlinks           = {true},
    linkcolor            = {black},
    urlcolor             = {blue},
    citecolor            = {black},
    filecolor            = {black},
}

\hbadness=10000	% Ignore underfull boxes

\begin{document}

\title{Herbstliches Kochbuch}
\author{Kathrin Welzel \& Marcel Gro{\ss}mann}
\maketitle

\begin{abstract}
    \noindent The examples in this document require at least version~1.4 of the \texttt{xcookybooky}\footnote{\url{http://www.ctan.org/pkg/xcookybooky}} package. For more examples and test recipes especially for using hook functions take a look at the source files located at \url{https://code.google.com/p/xcookybooky/}. If you are interested in modifying the layout of \texttt{xcookybooky} you will find examples in the documentation as well as in the configuration file \textbf{xcookybooky.cfg}.
\end{abstract}

\tableofcontents

\vspace{5em}

%\section{Recipes}
%The following recipes are examples for the usage of the \texttt{xcookybooky} package. The copyright of the pictures is owned by Roman Gaus. If you are using MiKTeX~2.9 you should get no errors, no warnings and no overfull boxes. The underfull boxes are suppressed due to the settings.

% background graphic
%\setBackgroundPicture[x, y=-2cm, width=\paperwidth-4cm, height, orientation = pagecenter]{pic/background}

\begin{otherlanguage}{ngerman}
\setHeadlines
{% translation
    inghead = Zutaten,
    prephead = Zubereitung,
    hinthead = Tipp,
    continuationhead = Fortsetzung,
    continuationfoot = Fortsetzung auf n\"achster Seite,
    portionvalue = Personen,
}

%%%%%%%%%%%%%%%%%%%%%%%%%%%%%%%%%%%%%%%%%%%%%%%%%%%%%%%%%%%%%%%%%%%
%				Recipe Section										%
%%%%%%%%%%%%%%%%%%%%%%%%%%%%%%%%%%%%%%%%%%%%%%%%%%%%%%%%%%%%%%%%%%%
%\newpage
\section{Vorspeisen}

\section{Hauptgerichte}

% Complete recipe example
\begin{recipe}
[% 
    preparationtime = {\unit[30]{min}},
    %bakingtime={\unit[1]{h}},
    %bakingtemperature={\protect\bakingtemperature{
     %   fanoven=\unit[230]{\textcelcius},
     %   topbottomheat=\unit[195]{°C},
     %   topheat=\unit[195]{°C},
     %   gasstove=Level 2}},
    portion = {\portion{5-6}},
    %calory={\unit[3]{kJ}},
    source = {Chefkoch}
]
{Grünkohl - Cremesüppchen}
    
    \graph
    {% pictures
        small=pic/CurlyKaleSoup2,     % small picture
        big=pic/CurlyKaleSoup  % big picture
    }
    
    %\introduction{%
     %   \blindtext
    %}
    
    \ingredients{%
        750 g	& 	Grünkohl\\
		350 g	& 	Kartoffeln\\
		1   	&	Zwiebel\\
		3 		&	Möhren\\
		1 Zehe	&	Knoblauch\\
		50 g	&	Butter\\
		1 EL	&	Cr{\`e}me fra{\^i}che\\
		1 Liter	&	kräftige Gemüsebrühe\\
 				&	Salz und Pfeffer\\
 				&	Muskat
    }
    
    \preparation{%
    \newline
        Den Grünkohl von den Rippen streifen, waschen, in Salzwasser blanchieren.
        Kartoffeln und Möhren schälen, grob würfeln. 
        Zwiebel würfeln und mit dem gehackten Knoblauch in Butter andünsten. Grünkohl, Möhren und Kartoffeln zugeben, kurz mitdünsten, dann Brühe zugeben und alles zusammen weich kochen lassen.
        Mit dem Pürierstab fein pürieren, Cr{\`e}me fra{\^i}che einrühren, dann mit Salz, Pfeffer und Muskat abschmecken.
        Man kann dazu Würstchen aller Art reichen.
        Eine schöne Variante ist auch, in die Teller ganz dünn geschnittenes, frisches Lachsfilet zu geben und die heiße Suppe dann darüber gießen.
    }
    
    %\suggestion[Headline]
    %{%
    %    \blindtext
   % }
    
   % \suggestion{%
    %    \blindtext
   % }
    
    \hint{%
        Die Grünkohlsuppe lässt sich mit Debrezinern gut ergänzen.
    }
    
\end{recipe}


\section{Nachspeisen}

\section{Kuchen}

\end{otherlanguage}


\end{document} 