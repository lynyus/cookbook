\begin{recipe}
[% 
    preparationtime = {\unit[30]{min}},
    portion = {\portion{4}},
    source = {Lieblingsrezept von Lisa}
]
{Ingwerlachs mit Blattspinat}
 
     \ingredients{%
        5 cm	& 	Ingwerwurzel \\
		4 EL	& 	Sojasoße \\
		   	&	Zucker \\
		6 EL	&	Sonnenblumenöl \\
			&	Pfeffer \\
		4	&	Lachsfilets mit Haut (à 200 g) \\
		1 Bund & Schnittlauch \\
		2 EL & Sesamsamen\\
		40 g & Butter\\
		1-2 EL & Weißweinessig\\
		1 EL & helle Miso-Paste\\
		450 g & junger Blattspinat\\
    }
    
    \preparation{%
    \newline
       \step Ingwer schälen und fein reiben. Ingwer, 3 EL Sojasauce, 1 TL Zucker, 1 EL Öl und etwas Pfeffer verrühren. Haut der Lachsfilets schräg einritzen. Filets mit der Marinade in einen Gefrierbeutel geben, gut verschließen und 10 Minuten marinieren.
       \step Schnittlauch in Röllchen schneiden. Sesamsamen in einer Pfanne ohne Fett anrösten. Butter in einem Topf zerlassen. 2 EL Öl, 50 ml Wasser, Essig, Miso-Paste und 1 Prise Zucker hinzufügen, aufkochen und mit dem Schneebesen verquirlen. vom Herd nehmen. Schnittlauch dazugeben, mit Pfeffer würzen. Miso-Butter warm halten.
       \step Lachsfilets aus der Marinade nehmen und abtropfen lassen. 1 EL Öl in einer beschichteten Pfanne erhitzen. Filets 4-6 Minuten auf der Hautseite braten. Wenden und 1 weitere Minute garen. Der Fisch sollte in der Mitte glasig sein.
       \step Restliches Öl (2 EL) in einem Topf erhitzen. Spinat darin zusammenfallen lassen. Mit der restlichen Sojasauce (1 EL) würzen. Auf Teller verteilen. Lachs auf den Spinat legen, mit Miso-Butter beträufeln. Mit gerösteten Sesam bestreuen.
    }
    

\end{recipe}