\begin{recipe}
[% 
    bakingtime={\unit[45]{min}},
    bakingtemperature={\protect\bakingtemperature{
        topbottomheat=\unit[160-170]{°C}}},
    portion = {\portion{2}},
    source = {Lieblingskuchen von Marie und Andreas}
]
{Sachertorte Originalrezept nach Carla Sacher}
   
   \ingredients{%
   		&\textit{1 dag = 10 g}\\
        28 dag	& 	Butter \\
	28 dag	& 	Zucker \\
		12&	Eier \\
		28 dag&	Schokolade \\
		22 dag & gesiebtes Mehl\\
		6 dag&Kakao\\
		&Marillenmarmelade\\
		&\textbf{Schokoladenglasur}\\
	75 dag& Schokolade\\
	3 dag& Kakao\\
	25 dag & Zucker oder Fondant\\
	ca. 125 ml& Wasser\\
 }
    
    \preparation{%
    \newline
       \step 28 dag Butter oder Rama mit 8 dag Zucker flaumig rühren.
       \step 12 Eidotter (bei kleinen Eiern mehr) unterrühren.
       \step 28 erweichte Schokolade – Achtung: nicht zu heiß, da sie sonst an Aroma
verliert – beigeben.
       \step 12 Eiklar zu Schnee schlagen, salzen, mit 20 dag Zucker zu festem Schnee
schlagen.
       \step 22 dag gesiebtes glattes Mehl und 6 dag Kakao unterrühren.
       \step Die Eisenringe mit Papier umwickeln und die Masse einfüllen.
       \step 45 Minuten bei 160-170 °C im Rohr backen.
       \step Erkalten lassen und aus den Ringen nehmen.
       \step In der Mitte durchschneiden, mit warmer Marillenmarmelade bestreichen und
zusammensetzen. Obere und äußere Seiten mit sehr heißer
Marillenmarmelade bestreichen.
\step 75 dag Schokolade zerkleinern.
\step 3 dag Kakao beigeben.
\step 25 dag Kristallzucker oder Fondant unterrühren.
\step 1/8 Liter Wasser (bei Bedarf etwas mehr) dazu leeren.
\step Köcheln lassen, bis kleine Fäden gezogen werden können.
\step Die Sachertorte mit der lippenwarmen Glasur übergießen.
    }
    
\end{recipe}