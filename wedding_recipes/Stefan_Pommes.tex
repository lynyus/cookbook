% Complete recipe example
\begin{recipe}
[% 
    preparationtime = {\unit[30]{min}},
    %bakingtime={\unit[1]{h}},
    %bakingtemperature={\protect\bakingtemperature{
     %   fanoven=\unit[230]{\textcelcius},
     %   topbottomheat=\unit[195]{°C},
     %   topheat=\unit[195]{°C},
     %   gasstove=Level 2}},
    portion = {\portion{4}},
    source = {Stefans traditionell-vegetarischer Hochgenuss}
]
{Selbstgemachte Pommes Frites (f{\"u}r den \glqq{}Profikoch\grqq{})}
    
    \graph
    {% pictures
        small=pic/Stefan_Pommes  % big picture
    }
    
    %\introduction{%
     %   \blindtext
    %}
    
    \ingredients{%
        4 EL	& 	Vollmeersalz \\
		32 Stk. 	& 	gute deutsche Kartoffeln  \\
		4 EL    &	Oliven{\"o}l (alternativ: Butter)\\
    }
    
    \preparation{%
    \newline
       \step Die Kartoffeln sch{\"a}len und in Stifte schneiden oder auf einem Gurkenschneider in Scheiben hobeln. 
       \step Dann ein Backblech einfetten und die Pommes darauf verteilen. Bei 200\textcelcius ~für ca. 15min auf der mittleren Schiene im Backofen backen. Einmal alles wenden und weitere 10min oder so lange backen, bis die Spitzen goldbraun und knusprig sind. Falls die Pommes labbrig sein sollen (nach traditioneller McDonalds-Art), die Zeiten um ca. 30-50\% verk{\"u}rzen.
       \step Vor dem Servieren mit etwas Vollmeersalz bestreuen.    
    }
    
    \suggestion[Servierbeispiel]
    {%
        Mit Ketchup bzw. Mayonaisse oder---f{\"u}r die ganz Wilden---mit beidem servieren (je nach pers{\"o}nlichem Geschmack).
    }
    
   % \suggestion{%
    %    \blindtext
   % }
    
    \hint{%
        Eine Nur-Pommes-Di{\"a}t f{\"u}hrt zu keinerlei Mangelern{\"a}hrung - also immer den Kartoffelkeller sch{\"o}n best{\"u}ckt halten, damit beliebige Krisen genussvoll ausgesessen werden k{\"o}nnen.
    }
    
\end{recipe}