% Complete recipe example
\begin{recipe}
[% 
    preparationtime = {\unit[30]{min}},
    %bakingtime={\unit[1]{h}},
    %bakingtemperature={\protect\bakingtemperature{
     %   fanoven=\unit[230]{\textcelcius},
     %   topbottomheat=\unit[195]{°C},
     %   topheat=\unit[195]{°C},
     %   gasstove=Level 2}},
    portion = {\portion{4}},
    %calory={\unit[3]{kJ}},
    source = {Vegetarisch Köstlich}
]
{Champignon-Zucchini-Pfanne}
    
    \graph
    {% pictures
        small=pic/ChampignonZucchiniPfanne,     % small picture
        big=pic/ChampignonZucchiniPfanne2  % big picture
    }
    
    %\introduction{%
     %   \blindtext
    %}
    
    \ingredients{%
     \textbf{Polenta:}			& 	 \\
    	750 ml					& 	Gemüsebrühe \\
		450 ml 					& 	Milch \\
		1	  					&	Lorbeerblatt \\
		180 g					&	Polenta\\
		30	g					&	Parmesan\\
		30	g					&	Butter\\
		\textbf{Pfanne:}		& 	 \\
		200 g					&	Zwiebeln \\
		1						&	Knoblauchzehe \\
		\unitfrac{1}{2} Bund	&	Petersilie \\
 		600 g					&	braune Champignons \\
 		3 EL					&	Olivenöl \\	
 		2 TL					&	Majoran \\
 		3 EL					&	dunkler Balsamico-Essig\\
 		 						&	Salz und Pfeffer
    }
    
    \preparation{%
    	\step Für die Polenta Brühe, Milch und Lorbeerblatt in Topf aufkochen. Polenta unter Rühren einstreuen, kurz aufkochen lassen und im geschlossenen Topf bei kleinster Hitze ca. 20 Minuten quellen lassen, dabei gelegentlich umrühren. Parmesan fein reiben.
    	\step In der Zwischenzeit für die Champignon-Pfanne Knoblauch und Zwiebeln schälen und würfeln. Petersilie abbrausen, trocken schütteln und fein hacken. Champignons putzen und halbieren oder vierteln. Zucchini waschen und fein würfeln.
    	\step Olivenöl in einer großen Pfanne erhitzen und Knoblauch, Zwiebeln darin ca. 3 Minuten anschwitzen. Champignons und Zuchhini dazugeben und ca. 5 Minuten braten. Majoran, Salz und Pfeffer würzen. Mit Balsamico ablöschen, Petersilie unterühren.
    	\step Polenta mit Salz und Pfeffer abschmecken, Lorbeerblatt entfernen, Butter einrühren. Parmesan unterühren. Polenta mit Gemüse servieren.   	
       	}
       
       
        \hint{%
       Die Polenta ist auch toll mit 1,2 l Gemüsebrühe, ohne Parmesan und mit 4 EL Olivenöl verfeinern.
   		}
       	

\end{recipe}