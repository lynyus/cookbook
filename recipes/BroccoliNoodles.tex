% Complete recipe example
\begin{recipe}
[% 
    preparationtime = {\unit[40]{min}},
    %bakingtime={\unit[1]{h}},
    %bakingtemperature={\protect\bakingtemperature{
     %   fanoven=\unit[230]{\textcelcius},
     %   topbottomheat=\unit[195]{°C},
     %   topheat=\unit[195]{°C},
     %   gasstove=Level 2}},
    portion = {\portion{4}},
    %calory={\unit[3]{kJ}},
    %source = {Vegetarisch Köstlich}
]
{Brokkoli Sugo}
    
    \graph
    {% pictures
       % small=pic/BroccoliNoodles2,     % small picture
        big=pic/BroccoliNoodles  % big picture
    }
    
    %\introduction{%
     %   \blindtext
    %}
    
    \ingredients{%
       500 g					& 	Brokkoli \\
		3 						& 	Zwiebeln \\
		1	  					&	Knoblauch \\
		300 g					&	Tomaten\\
		5						&	getrocknete Tomaten\\
			 					&	Kapern \\
		2 EL					&	Öl \\
		2 EL					&	Tomatenmark \\
		\unitfrac{1}{2} TL		&	Zucker \\
 		1 TL					&	Paprikapulver \\
 		\unitfrac{1}{2}			&	Limette \\	
 		300 g					&	Nudeln
    }
    
    \preparation{%
    	\step Brokkoli putzen, in kleine Röschen zerteilen, dicke Stiele schälen und in kleine Stücke schneiden, alles waschen. Salzwasser in Topf zum Kochen bringen und Stiele darin 3 Minuten blanchieren. Die Hälfte der Röschen zugeben und weitere 3 Minuten blanchieren danach abgießen und gut abtropfen lassen.
    	\step In der Zwischenzeit Knoblauch und Zwiebeln schälen und würfeln. Tomaten, getrocknete Tomaten würfeln. Öl in einem Topf erhitzen und Knoblauch und Zwiebeln unter rühren anbraten und mit Gemüsebrühe ablöschen. Restlichen Brokkoli, Tomaten, Kapern hinzugeben bei mittlerer Hitze zugedeck weichgaren.
    	\step In der Zwischenzeit Nudeln nach Packungsanleitung zubereiten.
    	\step Brokkoli Sugo mit dem Pürierstab pürieren, mit Gewürzenund Limettensaft abschmecken. Nudeln abgießen mit Sugo und Brokkoliröschen mischen und servieren.
   	
       	}
       
       
        \hint{%
        Gerne noch mit Parmesan oder Morzarella bestreuen. Alternativ: Cashewkerne im Mixer hacken und mit Hefeflocken mischen. 
   		}
       	

\end{recipe}