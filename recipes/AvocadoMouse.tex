% Complete recipe example
\begin{recipe}
[% 
    preparationtime = {\unit[80]{min}},
    %bakingtime={\unit[1]{h}},
    %bakingtemperature={\protect\bakingtemperature{
     %   fanoven=\unit[230]{\textcelcius},
     %   topbottomheat=\unit[195]{°C},
     %   topheat=\unit[195]{°C},
     %   gasstove=Level 2}},
    portion = {\portion{2-3}},
    %calory={\unit[3]{kJ}},
    source = {Vegetarisch Köstlich}
]
{Avocado-Mousse au Chocolat:}
    
    \graph
    {% pictures
       % small=pic/CurlyKaleSoup2,     % small picture
        big=pic/AvocadoMouse  % big picture
    }
    
    %\introduction{%
     %   \blindtext
    %}
    
    \ingredients{%
        2		& 	Avocado \\
		1		& 	Banane \\
		5 EL 	&	Kakao \\
		1 EL 	&	Agavendicksaft\\
    }
    
    \preparation{%
       \step Avocados schälen und Kern entfernen. Avocados grob kleinschneiden und in einen Mixer oder eine Schüssel geben.
       \step Kakao und Agavendicksaft hinzugeben.
       \step Alles im Mixer oder mit dem Stabmixer pürieren.
   
    }
    
    \hint
    {%
        Eine andere Variante benutzt Cashewkerne: 1 Tasse Cashewkerne mindestens 1 Stunde einweichen, Kerne nach dem Einweichen in den Mixer und mit 1 TL Agavendicksaft, 4 TL Kokosöl, \unitfrac{1}{2} TL Vanille, etwas Wasser und 3 TL Kakao mixen.
    }
    
   % \suggestion{%
    %    \blindtext
   % }
    
    
\end{recipe}