% Complete recipe example
\begin{recipe}
[% 
    preparationtime = {\unit[50]{min}},
    %bakingtime={\unit[1]{h}},
    %bakingtemperature={\protect\bakingtemperature{
     %   fanoven=\unit[230]{\textcelcius},
     %   topbottomheat=\unit[195]{°C},
     %   topheat=\unit[195]{°C},
     %   gasstove=Level 2}},
    portion = {\portion{4}},
    %calory={\unit[3]{kJ}},
    source = {Köstlich Vegetarisch}
]
{Indischer-Linsen-Fenchel-Eintopf}
    
    \graph
    {% pictures
        %small=pic/CurlyKaleSoup2,     % small picture
        big=pic/LinsenFenchelEintopf  % big picture
    }
    
    %\introduction{%
     %   \blindtext
    %}
    
    \ingredients{%
        700 g	& 	Fenchel\\
		250 g	& 	Möhren\\
		1   	&	rote Chilischote\\
		2 		&	walnussgroße Stücke Ingwer\\
		2 EL	&	Bratöl\\
		2 EL	&	gelbe Senfkörner\\
		2 EL	&	Kurkuma\\
		2 EL	&	Curry\\
 		200 g	&	Berglinsen\\
 		\unitfrac{1}{2} TL&	Zimt\\
 		200 g	&	Basmatireis\\
 	\unitfrac{1}{2} Bund & Petersilie\\ 	
 				&	Salz und Pfeffer\\
    }
    
    \preparation{%
    \step Fenchel waschen, putzen, Fenchelgrün zur Seite legen, Knollen vierteln, Strunk entfernen und in Viertel würfeln. Möhren ebenfalls würfeln. Chili waschen, der Länge nach halbieren, entkernen und fein würfeln. Ingwer schälen und klein hacken.
    \step Öl und Senfkörner in einem großen Topf leicht erhitzen, dabei Deckel darauf legen, die Senfkörner können springen. Ingwer und Chili zugeben und 1 Minute braten. Gemüse, Kurkuma und Curry zugeben, weitere 3 Minuten braten. Linsen und rund 600 ml Wasser zugeben. Zum Kochen bringen und im geschlossenem Topf auf kleiner Hitze ca. 25 Minuten garen, bis Linsen weich aber noch bissfest sind. Mit Salz und Zimt würzen und 3-4 Minuten durchziehen lassen.
    \step Inzwischen Reis nach Packungsanleitung (in der Regel mit doppelter Menge Wasser) gar kochen. Fenchelgrün und Petersilie abbrausen, trocken schütteln und fein hacken. Linsen-Fenchel-Eintopf mit Reis und Kräutern garniert servieren.
    }
    
    %\suggestion[Headline]
    %{%
    %    \blindtext
   % }
    
   % \suggestion{%
    %    \blindtext
   % }
    
    \hint{%
       Auch lecker: Am Ende noch ein paar Organgenscheiben unterheben und Curry mit Orangensaft abschmecken.
    }
    
\end{recipe}