% Complete recipe example
\begin{recipe}
[% 
    preparationtime = {\unit[30]{min}},
    %bakingtime={\unit[1]{h}},
    %bakingtemperature={\protect\bakingtemperature{
     %   fanoven=\unit[230]{\textcelcius},
     %   topbottomheat=\unit[195]{°C},
     %   topheat=\unit[195]{°C},
     %   gasstove=Level 2}},
    portion = {\portion{3-4}},
    %calory={\unit[3]{kJ}},
    source = {Wok}
]
{Weißkohlwok mit Ananas und Paprika}
    
    \graph
    {% pictures
       % small=pic/CurlyKaleSoup2,     % small picture
        big=pic/ColeSlawWok  % big picture
    }
    
    %\introduction{%
     %   \blindtext
    %}
    
    \ingredients{%
        500 g		& 	Weißkohl \\
		1			& 	rote Paprika \\
		200 g  		&	Ananas \\
		1 Stängel	&	Zitronengras \\
		1 			&	Chilischote \\
		200	g		&	(Mungobohnen-) Sprossen \\
		2 EL		&	Sojasauce \\
		4 EL		&	Sherry \\
 					&	Salz und Pfeffer \\
 		1 TL		&	Speisestärke \\
 		2 EL		& 	Kokosraspel	\\
 		1 EL 		& 	weiße Sesamsamen \\
 		\unitfrac{1}{2} TL & Zitronenschale \\
 		4 EL		& 	Öl 
    }
    
    \preparation{%
       \step Den Kohl und die Paprika waschen, putzen und in feine Streifen schneiden. Die Ananas schälen, den Strunk entfernen und in Stücke schneiden. Zitronengras putzen, waschen und die Zwiebel hacken. Chilischote halbieren, entkernen und in feine Scheiben schneiden. Sprossen in einem Sieb abbrausen und abtropfen lassen.
       \step Sojasauce, Sherry, Salz und Pfeffer zu einer Sauce verquirlen.
       \step Kokosraspeln und Sesam im Wok ohne Fett goldgelb rösten.
       \step Den Wok mit Öl erhitzen. Kohl- und Paprikstreifen unter Rühren bissfest garen. Ananas, Zitronengras, Chilischote und Sprossen hinzugeben; die Saucenmischung unterrühren. Etwa 2 Minuten garen lassen und mit Kokosraspel-Sesam-Mischung und Zitronenschalen bestreuen.
   
    }
    
    %\suggestion[Headline]
    %{%
    %    \blindtext
   % }
    
   % \suggestion{%
    %    \blindtext
   % }
    
    \hint{%
        Eine Messerspitze Bockshornkleesamen macht das Gericht bekömmlicher.
    }
    
\end{recipe}