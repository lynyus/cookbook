% Complete recipe example
\begin{recipe}
[% 
    preparationtime = {\unit[30]{min}},
    %bakingtime={\unit[1]{h}},
    %bakingtemperature={\protect\bakingtemperature{
     %   fanoven=\unit[230]{\textcelcius},
     %   topbottomheat=\unit[195]{°C},
     %   topheat=\unit[195]{°C},
     %   gasstove=Level 2}},
    portion = {\portion{5-6}},
    %calory={\unit[3]{kJ}},
    source = {Chefkoch}
]
{Grünkohl - Cremesüppchen}
    
    \graph
    {% pictures
        small=pic/CurlyKaleSoup2,     % small picture
        big=pic/CurlyKaleSoup  % big picture
    }
    
    %\introduction{%
     %   \blindtext
    %}
    
    \ingredients{%
        750 g	& 	Grünkohl\\
		350 g	& 	Kartoffeln\\
		1   	&	Zwiebel\\
		3 		&	Möhren\\
		1 Zehe	&	Knoblauch\\
		50 g	&	Butter\\
		1 EL	&	Cr{\`e}me fra{\^i}che\\
		1 Liter	&	kräftige Gemüsebrühe\\
 				&	Salz und Pfeffer\\
 				&	Muskat
    }
    
    \preparation{%
       \step Den Grünkohl von den Rippen streifen, waschen, in Salzwasser blanchieren.
       \step Kartoffeln und Möhren schälen, grob würfeln. Zwiebel ebenfalls würfeln, den Knoblauch hacken.
       \step Die Zwiebeln mit dem gehackten Knoblauch in Butter andünsten. Grünkohl, Möhren und Kartoffeln zugeben, kurz mitdünsten, dann Brühe zugeben und alles zusammen weich kochen lassen.
       \step Mit dem Pürierstab fein pürieren, Cr{\`e}me fra{\^i}che einrühren, dann mit Salz, Pfeffer und Muskat abschmecken.
   
    }
    
    %\suggestion[Headline]
    %{%
    %    \blindtext
   % }
    
   % \suggestion{%
    %    \blindtext
   % }
    
    \hint{%
        Die Grünkohlsuppe lässt sich mit Debrezinern (Würstchen) gut ergänzen. Eine schöne Variante ist auch, in die Teller ganz dünn geschnittenes, frisches Lachsfilet zu geben und die heiße Suppe dann darüber gießen.
    }
    
\end{recipe}