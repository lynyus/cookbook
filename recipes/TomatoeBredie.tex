% Complete recipe example
\begin{recipe}
[% 
    preparationtime = {\unit[80]{min}},
    %bakingtime={\unit[1]{h}},
    %bakingtemperature={\protect\bakingtemperature{
     %   fanoven=\unit[230]{\textcelcius},
     %   topbottomheat=\unit[195]{°C},
     %   topheat=\unit[195]{°C},
     %   gasstove=Level 2}},
    portion = {\portion{4}},
    %calory={\unit[3]{kJ}},
    source = {Vegetarisch Köstlich}
]
{Tomatentopf mit Seitanwürfeln und gelbem Würzreis}
    
    \graph
    {% pictures
       % small=pic/CurlyKaleSoup2,     % small picture
        big=pic/TomatoeBredie  % big picture
    }
    
    %\introduction{%
     %   \blindtext
    %}
    
    \ingredients{%
        100 g		& 	Zwiebeln \\
		3 Zehen		& 	Knoblauch \\
		300 g  		&	festkochenende Kartoffeln \\
		500 g		&	Tomaten\\
		200 g		&	weiße Bohnen \\
		1 Bund 		&	Majoran \\
		3 EL		&	Öl \\
		2 EL		&	Tomatenmark \\
		1 TL		&	Zucker \\
 		1 TL		&	Paprikapulver \\
 		1 Msp		& 	Chilipulver	\\
 		350 ml 		& 	Gemüsebrühe \\
 		200 g		&	Seitan \\
 		200 g		& paraboiled Reis\\
 		60 g		& Rosinen\\
 		1 TL		& Kurkuma\\
 		1			& Zimtstange
    }
    
    \preparation{%
       \step Zwiebeln und Knoblauch fein würfeln. Kartoffeln schälen, Tomaten waschen und beide 1\unitfrac{1}{2}cm groß würfeln. Öl in großem Topf erhitzen, Zwiebeln darin bei mittlerer Hitze ca. 4 Minuten goldgelb anrbaten. Tomatenmark und Knoblauch zugeben und 3 Minuten unter Rühren mitbraten. Tomaten, Kartoffeln, Bohnen, Majoran,  Zucker, 1TL Salz, Paprikapulver, Chilipulver und Gemüsebrühe hinzugeben und 50 Minuten bei mittlerer Hitze im offenen Topf dicklich einkochen lassen. Zwischendurch umrühren.
       \step Inzwischen den Reis nach Packungsanleitung mit Rosinen, Kurkuma, Zimtstange und Salz zubereiten. Ggf. restliche Flüssigkeit abgießen und Zimstange entfernen.
       \step Seitan 1cm groß würfeln und in einer Pfanne mit restlichem Öl ca. 5-7 Minuten kross anbraten. Zum Tomatentopf geben und alles mit Salz und Pfeffer anbraten.
   
    }
    
    \suggestion
    {%
           Traditionell wird dieser Topf mit Lamm oder Hammelfleisch gekocht und ist ein südafrikanisches Gericht. Je länger der Eintopf simmert und zieht desto intensiver schmeckt er, hierfür kann man auch getrocknete Bohnen verwenden und die Kochzeit dementsprechend erhöhen. 
    }
    
   % \suggestion{%
    %    \blindtext
   % }
    
    
\end{recipe}