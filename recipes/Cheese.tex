% Complete recipe example
\begin{recipe}
[% 
    preparationtime = {\unit[30]{min}},
    bakingtime={\unit[10]{min}},
    %bakingtemperature={\protect\bakingtemperature{
     %   fanoven=\unit[230]{\textcelcius},
     %   topbottomheat=\unit[195]{°C},
     %   topheat=\unit[195]{°C},
     %   gasstove=Level 2}},
    portion = {\portion{5-6}},
    %calory={\unit[3]{kJ}},
    source = {Chefkoch}
]
{Käsefondue}
    
    \graph
    {% pictures
        small=pic/Cheese,     % small picture
        big=pic/AmelieCheese  % big picture
    }
    
    %\introduction{%
     %   \blindtext
    %}
    
    \ingredients{%
    	4						& 	Baguette \\
    	300 g					& 	Appenzeller \\
    	300 g					& 	Bergkäse \\
    	300 g					& 	Emmentaler \\
      	300 g					& 	Gruy{\'e}re \\
		1 Flasche 				&	Weißwein \\
		2 TL 	 				&	Speisestärke \\
		\unitfrac{1}{2} TL		&	Natron \\
		4 cl					&	Kirschwasser \\
								&	Pfeffer \\
		2 Msp.					& 	Paprikapulver \\
		2 Msp.					&	Muskat \\
		1 Zehe					&	Knoblauch \\
    }
    
    \preparation{%
    \step Den Fonduetopf mit der einen Hälfte des Knoblauchs ausreiben.
    \step Wein in den Topf gießen und bei kleiner Hitze langsam erwärmen. 
    \step Ist der Wein heiß, den Käse portionsweise hineingeben und unter ständigem Rühren schmelzen lassen.
    \step Die andere Hälfte der Knoblauchzehe pressen und zum Käse geben. 
    \step Die Speisestärke mit dem Kirschwasser verrühren, zum Käse geben und unter Rühren noch mal aufkochen lassen. Das Fondue mit Pfeffer, Paprika und Muskat würzen und Natron hinzugeben.
    }
    
    \suggestion[Am Tisch]{%
        Die Flamme im Rechaud entzünden.	
		Baguettestücke auf Fonduegabeln spießen, in das Fondue eintauchen und genießen.   
    }
    
   % \suggestion{%
    %    \blindtext
   % }
    
    \hint{%
        Zusätzlich zu Baguette können auch Weintrauben, Ananas und Paprika gereicht werden.
    }
    
\end{recipe}